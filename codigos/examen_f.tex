% Options for packages loaded elsewhere
\PassOptionsToPackage{unicode}{hyperref}
\PassOptionsToPackage{hyphens}{url}
%
\documentclass[
]{article}
\usepackage{amsmath,amssymb}
\usepackage{lmodern}
\usepackage{iftex}
\ifPDFTeX
  \usepackage[T1]{fontenc}
  \usepackage[utf8]{inputenc}
  \usepackage{textcomp} % provide euro and other symbols
\else % if luatex or xetex
  \usepackage{unicode-math}
  \defaultfontfeatures{Scale=MatchLowercase}
  \defaultfontfeatures[\rmfamily]{Ligatures=TeX,Scale=1}
\fi
% Use upquote if available, for straight quotes in verbatim environments
\IfFileExists{upquote.sty}{\usepackage{upquote}}{}
\IfFileExists{microtype.sty}{% use microtype if available
  \usepackage[]{microtype}
  \UseMicrotypeSet[protrusion]{basicmath} % disable protrusion for tt fonts
}{}
\makeatletter
\@ifundefined{KOMAClassName}{% if non-KOMA class
  \IfFileExists{parskip.sty}{%
    \usepackage{parskip}
  }{% else
    \setlength{\parindent}{0pt}
    \setlength{\parskip}{6pt plus 2pt minus 1pt}}
}{% if KOMA class
  \KOMAoptions{parskip=half}}
\makeatother
\usepackage{xcolor}
\usepackage[margin=1in]{geometry}
\usepackage{color}
\usepackage{fancyvrb}
\newcommand{\VerbBar}{|}
\newcommand{\VERB}{\Verb[commandchars=\\\{\}]}
\DefineVerbatimEnvironment{Highlighting}{Verbatim}{commandchars=\\\{\}}
% Add ',fontsize=\small' for more characters per line
\usepackage{framed}
\definecolor{shadecolor}{RGB}{248,248,248}
\newenvironment{Shaded}{\begin{snugshade}}{\end{snugshade}}
\newcommand{\AlertTok}[1]{\textcolor[rgb]{0.94,0.16,0.16}{#1}}
\newcommand{\AnnotationTok}[1]{\textcolor[rgb]{0.56,0.35,0.01}{\textbf{\textit{#1}}}}
\newcommand{\AttributeTok}[1]{\textcolor[rgb]{0.77,0.63,0.00}{#1}}
\newcommand{\BaseNTok}[1]{\textcolor[rgb]{0.00,0.00,0.81}{#1}}
\newcommand{\BuiltInTok}[1]{#1}
\newcommand{\CharTok}[1]{\textcolor[rgb]{0.31,0.60,0.02}{#1}}
\newcommand{\CommentTok}[1]{\textcolor[rgb]{0.56,0.35,0.01}{\textit{#1}}}
\newcommand{\CommentVarTok}[1]{\textcolor[rgb]{0.56,0.35,0.01}{\textbf{\textit{#1}}}}
\newcommand{\ConstantTok}[1]{\textcolor[rgb]{0.00,0.00,0.00}{#1}}
\newcommand{\ControlFlowTok}[1]{\textcolor[rgb]{0.13,0.29,0.53}{\textbf{#1}}}
\newcommand{\DataTypeTok}[1]{\textcolor[rgb]{0.13,0.29,0.53}{#1}}
\newcommand{\DecValTok}[1]{\textcolor[rgb]{0.00,0.00,0.81}{#1}}
\newcommand{\DocumentationTok}[1]{\textcolor[rgb]{0.56,0.35,0.01}{\textbf{\textit{#1}}}}
\newcommand{\ErrorTok}[1]{\textcolor[rgb]{0.64,0.00,0.00}{\textbf{#1}}}
\newcommand{\ExtensionTok}[1]{#1}
\newcommand{\FloatTok}[1]{\textcolor[rgb]{0.00,0.00,0.81}{#1}}
\newcommand{\FunctionTok}[1]{\textcolor[rgb]{0.00,0.00,0.00}{#1}}
\newcommand{\ImportTok}[1]{#1}
\newcommand{\InformationTok}[1]{\textcolor[rgb]{0.56,0.35,0.01}{\textbf{\textit{#1}}}}
\newcommand{\KeywordTok}[1]{\textcolor[rgb]{0.13,0.29,0.53}{\textbf{#1}}}
\newcommand{\NormalTok}[1]{#1}
\newcommand{\OperatorTok}[1]{\textcolor[rgb]{0.81,0.36,0.00}{\textbf{#1}}}
\newcommand{\OtherTok}[1]{\textcolor[rgb]{0.56,0.35,0.01}{#1}}
\newcommand{\PreprocessorTok}[1]{\textcolor[rgb]{0.56,0.35,0.01}{\textit{#1}}}
\newcommand{\RegionMarkerTok}[1]{#1}
\newcommand{\SpecialCharTok}[1]{\textcolor[rgb]{0.00,0.00,0.00}{#1}}
\newcommand{\SpecialStringTok}[1]{\textcolor[rgb]{0.31,0.60,0.02}{#1}}
\newcommand{\StringTok}[1]{\textcolor[rgb]{0.31,0.60,0.02}{#1}}
\newcommand{\VariableTok}[1]{\textcolor[rgb]{0.00,0.00,0.00}{#1}}
\newcommand{\VerbatimStringTok}[1]{\textcolor[rgb]{0.31,0.60,0.02}{#1}}
\newcommand{\WarningTok}[1]{\textcolor[rgb]{0.56,0.35,0.01}{\textbf{\textit{#1}}}}
\usepackage{graphicx}
\makeatletter
\def\maxwidth{\ifdim\Gin@nat@width>\linewidth\linewidth\else\Gin@nat@width\fi}
\def\maxheight{\ifdim\Gin@nat@height>\textheight\textheight\else\Gin@nat@height\fi}
\makeatother
% Scale images if necessary, so that they will not overflow the page
% margins by default, and it is still possible to overwrite the defaults
% using explicit options in \includegraphics[width, height, ...]{}
\setkeys{Gin}{width=\maxwidth,height=\maxheight,keepaspectratio}
% Set default figure placement to htbp
\makeatletter
\def\fps@figure{htbp}
\makeatother
\setlength{\emergencystretch}{3em} % prevent overfull lines
\providecommand{\tightlist}{%
  \setlength{\itemsep}{0pt}\setlength{\parskip}{0pt}}
\setcounter{secnumdepth}{-\maxdimen} % remove section numbering
\ifLuaTeX
  \usepackage{selnolig}  % disable illegal ligatures
\fi
\IfFileExists{bookmark.sty}{\usepackage{bookmark}}{\usepackage{hyperref}}
\IfFileExists{xurl.sty}{\usepackage{xurl}}{} % add URL line breaks if available
\urlstyle{same} % disable monospaced font for URLs
\hypersetup{
  pdftitle={examen\_f.R},
  pdfauthor={Usuario},
  hidelinks,
  pdfcreator={LaTeX via pandoc}}

\title{examen\_f.R}
\author{Usuario}
\date{2023-11-29}

\begin{document}
\maketitle

\begin{Shaded}
\begin{Highlighting}[]
\CommentTok{\#LUZ ELENA RODRÍGUEZ PEQUEÑO}
\CommentTok{\#2070472}
\CommentTok{\#29/11/2023}
\CommentTok{\#Examen final}

\CommentTok{\# importar {-}{-}{-}{-}{-}{-}{-}{-}{-}{-}{-}{-}{-}{-}{-}{-}{-}{-}{-}{-}{-}{-}{-}{-}{-}{-}{-}{-}{-}{-}{-}{-}{-}{-}{-}{-}{-}{-}{-}{-}{-}{-}{-}{-}{-}{-}{-}{-}{-}{-}{-}{-}{-}{-}{-}{-}{-}{-}{-}{-}{-}{-}{-}{-}}

\FunctionTok{setwd}\NormalTok{(}\StringTok{"C:/Repositorio\_LR/Met\_ES/codigos"}\NormalTok{)}
\NormalTok{madera }\OtherTok{\textless{}{-}} \FunctionTok{read.csv}\NormalTok{(}\StringTok{"madera.csv"}\NormalTok{, }\AttributeTok{header =} \ConstantTok{TRUE}\NormalTok{) }
\FunctionTok{head}\NormalTok{(madera)}
\end{Highlighting}
\end{Shaded}

\begin{verbatim}
##   Encino Pino
## 1   16.6 12.6
## 2   16.8 14.4
## 3   17.2 12.6
## 4   17.6 12.0
## 5   17.2 13.2
## 6   18.6 13.2
\end{verbatim}

\begin{Shaded}
\begin{Highlighting}[]
\CommentTok{\# PARTE 1 {-}{-}{-}{-}{-}{-}{-}{-}{-}{-}{-}{-}{-}{-}{-}{-}{-}{-}{-}{-}{-}{-}{-}{-}{-}{-}{-}{-}{-}{-}{-}{-}{-}{-}{-}{-}{-}{-}{-}{-}{-}{-}{-}{-}{-}{-}{-}{-}{-}{-}{-}{-}{-}{-}{-}{-}{-}{-}{-}{-}{-}{-}{-}{-}{-}}



\CommentTok{\# descriptivas {-}{-}{-}{-}{-}{-}{-}{-}{-}{-}{-}{-}{-}{-}{-}{-}{-}{-}{-}{-}{-}{-}{-}{-}{-}{-}{-}{-}{-}{-}{-}{-}{-}{-}{-}{-}{-}{-}{-}{-}{-}{-}{-}{-}{-}{-}{-}{-}{-}{-}{-}{-}{-}{-}{-}{-}{-}{-}{-}{-}}

\FunctionTok{mean}\NormalTok{(madera}\SpecialCharTok{$}\NormalTok{Encino) }\CommentTok{\#17.46}
\end{Highlighting}
\end{Shaded}

\begin{verbatim}
## [1] 17.46
\end{verbatim}

\begin{Shaded}
\begin{Highlighting}[]
\FunctionTok{median}\NormalTok{(madera}\SpecialCharTok{$}\NormalTok{Encino) }\CommentTok{\#17.3}
\end{Highlighting}
\end{Shaded}

\begin{verbatim}
## [1] 17.3
\end{verbatim}

\begin{Shaded}
\begin{Highlighting}[]
\FunctionTok{range}\NormalTok{(madera}\SpecialCharTok{$}\NormalTok{Encino) }
\end{Highlighting}
\end{Shaded}

\begin{verbatim}
## [1] 16.2 19.0
\end{verbatim}

\begin{Shaded}
\begin{Highlighting}[]
\FunctionTok{mean}\NormalTok{(madera}\SpecialCharTok{$}\NormalTok{Pino) }\CommentTok{\#12.68}
\end{Highlighting}
\end{Shaded}

\begin{verbatim}
## [1] 12.68
\end{verbatim}

\begin{Shaded}
\begin{Highlighting}[]
\FunctionTok{median}\NormalTok{(madera}\SpecialCharTok{$}\NormalTok{Pino) }\CommentTok{\#12.5}
\end{Highlighting}
\end{Shaded}

\begin{verbatim}
## [1] 12.5
\end{verbatim}

\begin{Shaded}
\begin{Highlighting}[]
\FunctionTok{range}\NormalTok{(madera}\SpecialCharTok{$}\NormalTok{Encino)}
\end{Highlighting}
\end{Shaded}

\begin{verbatim}
## [1] 16.2 19.0
\end{verbatim}

\begin{Shaded}
\begin{Highlighting}[]
\CommentTok{\# grafica {-}{-}{-}{-}{-}{-}{-}{-}{-}{-}{-}{-}{-}{-}{-}{-}{-}{-}{-}{-}{-}{-}{-}{-}{-}{-}{-}{-}{-}{-}{-}{-}{-}{-}{-}{-}{-}{-}{-}{-}{-}{-}{-}{-}{-}{-}{-}{-}{-}{-}{-}{-}{-}{-}{-}{-}{-}{-}{-}{-}{-}{-}{-}{-}{-}}

\FunctionTok{boxplot}\NormalTok{(madera}\SpecialCharTok{$}\NormalTok{Encino, madera}\SpecialCharTok{$}\NormalTok{Pino , }\AttributeTok{xlab =} \StringTok{"encino"}\NormalTok{, }
        \AttributeTok{ylab =} \StringTok{"pino"}\NormalTok{, }\AttributeTok{pch =} \DecValTok{19}\NormalTok{)}
\end{Highlighting}
\end{Shaded}

\includegraphics{examen_f_files/figure-latex/unnamed-chunk-1-1.pdf}

\begin{Shaded}
\begin{Highlighting}[]
\CommentTok{\# PARTE 2 {-}{-}{-}{-}{-}{-}{-}{-}{-}{-}{-}{-}{-}{-}{-}{-}{-}{-}{-}{-}{-}{-}{-}{-}{-}{-}{-}{-}{-}{-}{-}{-}{-}{-}{-}{-}{-}{-}{-}{-}{-}{-}{-}{-}{-}{-}{-}{-}{-}{-}{-}{-}{-}{-}{-}{-}{-}{-}{-}{-}{-}{-}{-}{-}{-}}


\CommentTok{\# hipotesis {-}{-}{-}{-}{-}{-}{-}{-}{-}{-}{-}{-}{-}{-}{-}{-}{-}{-}{-}{-}{-}{-}{-}{-}{-}{-}{-}{-}{-}{-}{-}{-}{-}{-}{-}{-}{-}{-}{-}{-}{-}{-}{-}{-}{-}{-}{-}{-}{-}{-}{-}{-}{-}{-}{-}{-}{-}{-}{-}{-}{-}{-}{-}}
\CommentTok{\#17.3 vs 12.5}
\CommentTok{\#la madera de la especie de encino pesa mas que la de pino segun los datos proporcionados }


\CommentTok{\# procedimiento {-}{-}{-}{-}{-}{-}{-}{-}{-}{-}{-}{-}{-}{-}{-}{-}{-}{-}{-}{-}{-}{-}{-}{-}{-}{-}{-}{-}{-}{-}{-}{-}{-}{-}{-}{-}{-}{-}{-}{-}{-}{-}{-}{-}{-}{-}{-}{-}{-}{-}{-}{-}{-}{-}{-}{-}{-}{-}{-}}
\FunctionTok{t.test}\NormalTok{(madera}\SpecialCharTok{$}\NormalTok{Encino, }\AttributeTok{mu =} \DecValTok{17}\NormalTok{)}
\end{Highlighting}
\end{Shaded}

\begin{verbatim}
## 
##  One Sample t-test
## 
## data:  madera$Encino
## t = 3.3907, df = 29, p-value = 0.002029
## alternative hypothesis: true mean is not equal to 17
## 95 percent confidence interval:
##  17.18254 17.73746
## sample estimates:
## mean of x 
##     17.46
\end{verbatim}

\begin{Shaded}
\begin{Highlighting}[]
\CommentTok{\#t = 3.3907, df = 29, p{-}value = 0.002029}
\FunctionTok{t.test}\NormalTok{(madera}\SpecialCharTok{$}\NormalTok{Encino, }\AttributeTok{mu =} \FloatTok{16.5}\NormalTok{)}
\end{Highlighting}
\end{Shaded}

\begin{verbatim}
## 
##  One Sample t-test
## 
## data:  madera$Encino
## t = 7.0763, df = 29, p-value = 8.743e-08
## alternative hypothesis: true mean is not equal to 16.5
## 95 percent confidence interval:
##  17.18254 17.73746
## sample estimates:
## mean of x 
##     17.46
\end{verbatim}

\begin{Shaded}
\begin{Highlighting}[]
\CommentTok{\#t = 7.0763, df = 29, p{-}value = 8.743e{-}08}
\FunctionTok{t.test}\NormalTok{(madera}\SpecialCharTok{$}\NormalTok{Encino, }\AttributeTok{mu =} \FloatTok{16.6}\NormalTok{)}
\end{Highlighting}
\end{Shaded}

\begin{verbatim}
## 
##  One Sample t-test
## 
## data:  madera$Encino
## t = 6.3392, df = 29, p-value = 6.308e-07
## alternative hypothesis: true mean is not equal to 16.6
## 95 percent confidence interval:
##  17.18254 17.73746
## sample estimates:
## mean of x 
##     17.46
\end{verbatim}

\begin{Shaded}
\begin{Highlighting}[]
\CommentTok{\#t = 6.3392, df = 29, p{-}value = 6.308e{-}07}
\FunctionTok{t.test}\NormalTok{(madera}\SpecialCharTok{$}\NormalTok{Encino, }\AttributeTok{mu =} \FloatTok{8.5}\NormalTok{)}
\end{Highlighting}
\end{Shaded}

\begin{verbatim}
## 
##  One Sample t-test
## 
## data:  madera$Encino
## t = 66.046, df = 29, p-value < 2.2e-16
## alternative hypothesis: true mean is not equal to 8.5
## 95 percent confidence interval:
##  17.18254 17.73746
## sample estimates:
## mean of x 
##     17.46
\end{verbatim}

\begin{Shaded}
\begin{Highlighting}[]
\CommentTok{\#t = 66.046, df = 29, p{-}value \textless{} 2.2e{-}16}

\FunctionTok{t.test}\NormalTok{(madera}\SpecialCharTok{$}\NormalTok{Pino, }\AttributeTok{mu =} \DecValTok{13}\NormalTok{)}
\end{Highlighting}
\end{Shaded}

\begin{verbatim}
## 
##  One Sample t-test
## 
## data:  madera$Pino
## t = -1.5971, df = 29, p-value = 0.1211
## alternative hypothesis: true mean is not equal to 13
## 95 percent confidence interval:
##  12.2702 13.0898
## sample estimates:
## mean of x 
##     12.68
\end{verbatim}

\begin{Shaded}
\begin{Highlighting}[]
\CommentTok{\#t = {-}1.5971, df = 29, p{-}value = 0.1211}
\FunctionTok{t.test}\NormalTok{(madera}\SpecialCharTok{$}\NormalTok{Pino, }\AttributeTok{mu =} \FloatTok{13.5}\NormalTok{)}
\end{Highlighting}
\end{Shaded}

\begin{verbatim}
## 
##  One Sample t-test
## 
## data:  madera$Pino
## t = -4.0925, df = 29, p-value = 0.000311
## alternative hypothesis: true mean is not equal to 13.5
## 95 percent confidence interval:
##  12.2702 13.0898
## sample estimates:
## mean of x 
##     12.68
\end{verbatim}

\begin{Shaded}
\begin{Highlighting}[]
\CommentTok{\#t = {-}4.0925, df = 29, p{-}value = 0.000311}
\FunctionTok{t.test}\NormalTok{(madera}\SpecialCharTok{$}\NormalTok{Pino, }\AttributeTok{mu =} \FloatTok{13.6}\NormalTok{)}
\end{Highlighting}
\end{Shaded}

\begin{verbatim}
## 
##  One Sample t-test
## 
## data:  madera$Pino
## t = -4.5916, df = 29, p-value = 7.882e-05
## alternative hypothesis: true mean is not equal to 13.6
## 95 percent confidence interval:
##  12.2702 13.0898
## sample estimates:
## mean of x 
##     12.68
\end{verbatim}

\begin{Shaded}
\begin{Highlighting}[]
\CommentTok{\#t = {-}4.5916, df = 29, p{-}value = 7.882e{-}05}
\FunctionTok{t.test}\NormalTok{(madera}\SpecialCharTok{$}\NormalTok{Pino, }\AttributeTok{mu =} \FloatTok{6.5}\NormalTok{)}
\end{Highlighting}
\end{Shaded}

\begin{verbatim}
## 
##  One Sample t-test
## 
## data:  madera$Pino
## t = 30.843, df = 29, p-value < 2.2e-16
## alternative hypothesis: true mean is not equal to 6.5
## 95 percent confidence interval:
##  12.2702 13.0898
## sample estimates:
## mean of x 
##     12.68
\end{verbatim}

\begin{Shaded}
\begin{Highlighting}[]
\CommentTok{\#t = 30.843, df = 29, p{-}value \textless{} 2.2e{-}16}


\CommentTok{\# recapibilidad {-}{-}{-}{-}{-}{-}{-}{-}{-}{-}{-}{-}{-}{-}{-}{-}{-}{-}{-}{-}{-}{-}{-}{-}{-}{-}{-}{-}{-}{-}{-}{-}{-}{-}{-}{-}{-}{-}{-}{-}{-}{-}{-}{-}{-}{-}{-}{-}{-}{-}{-}{-}{-}{-}{-}{-}{-}{-}{-}}

\CommentTok{\#Guardar la prueba t en un objeto llamado "prueba"}
\NormalTok{prueba }\OtherTok{\textless{}{-}} \FunctionTok{t.test}\NormalTok{(madera}\SpecialCharTok{$}\NormalTok{Encino, }\AttributeTok{mu =}\DecValTok{17}\NormalTok{)}

\CommentTok{\#Conocer el p{-}value}
\NormalTok{prueba}\SpecialCharTok{$}\NormalTok{p.value }
\end{Highlighting}
\end{Shaded}

\begin{verbatim}
## [1] 0.002028846
\end{verbatim}

\begin{Shaded}
\begin{Highlighting}[]
\CommentTok{\#0.002028846}

\CommentTok{\# Conocer los grados de libertad }
\NormalTok{prueba}\SpecialCharTok{$}\NormalTok{parameter}
\end{Highlighting}
\end{Shaded}

\begin{verbatim}
## df 
## 29
\end{verbatim}

\begin{Shaded}
\begin{Highlighting}[]
\CommentTok{\#df 29}

\CommentTok{\# Conocer intervalos de confianza }
\NormalTok{prueba}\SpecialCharTok{$}\NormalTok{conf.int}
\end{Highlighting}
\end{Shaded}

\begin{verbatim}
## [1] 17.18254 17.73746
## attr(,"conf.level")
## [1] 0.95
\end{verbatim}

\begin{Shaded}
\begin{Highlighting}[]
\CommentTok{\#17.18254 17.73746 \#0.95}


\CommentTok{\# PARTE 3 {-}{-}{-}{-}{-}{-}{-}{-}{-}{-}{-}{-}{-}{-}{-}{-}{-}{-}{-}{-}{-}{-}{-}{-}{-}{-}{-}{-}{-}{-}{-}{-}{-}{-}{-}{-}{-}{-}{-}{-}{-}{-}{-}{-}{-}{-}{-}{-}{-}{-}{-}{-}{-}{-}{-}{-}{-}{-}{-}{-}{-}{-}{-}{-}{-}}

\CommentTok{\# importar {-}{-}{-}{-}{-}{-}{-}{-}{-}{-}{-}{-}{-}{-}{-}{-}{-}{-}{-}{-}{-}{-}{-}{-}{-}{-}{-}{-}{-}{-}{-}{-}{-}{-}{-}{-}{-}{-}{-}{-}{-}{-}{-}{-}{-}{-}{-}{-}{-}{-}{-}{-}{-}{-}{-}{-}{-}{-}{-}{-}{-}{-}{-}{-}}

\FunctionTok{setwd}\NormalTok{(}\StringTok{"C:/Repositorio\_LR/Met\_ES/codigos"}\NormalTok{)}
\NormalTok{emiciones }\OtherTok{\textless{}{-}} \FunctionTok{read.csv}\NormalTok{(}\StringTok{"emiciones.csv"}\NormalTok{, }\AttributeTok{header =} \ConstantTok{TRUE}\NormalTok{) }
\FunctionTok{head}\NormalTok{(emiciones)}
\end{Highlighting}
\end{Shaded}

\begin{verbatim}
##   azufre
## 1   15.8
## 2   22.7
## 3   26.8
## 4   19.1
## 5   18.5
## 6   14.4
\end{verbatim}

\begin{Shaded}
\begin{Highlighting}[]
\FunctionTok{mean}\NormalTok{(emiciones}\SpecialCharTok{$}\NormalTok{azufre) }\CommentTok{\#18.7075}
\end{Highlighting}
\end{Shaded}

\begin{verbatim}
## [1] 18.7075
\end{verbatim}

\begin{Shaded}
\begin{Highlighting}[]
\FunctionTok{median}\NormalTok{(emiciones}\SpecialCharTok{$}\NormalTok{azufre)}\CommentTok{\#18.8}
\end{Highlighting}
\end{Shaded}

\begin{verbatim}
## [1] 18.8
\end{verbatim}

\begin{Shaded}
\begin{Highlighting}[]
\FunctionTok{t.test}\NormalTok{(emiciones}\SpecialCharTok{$}\NormalTok{azufre, }\AttributeTok{mu =} \DecValTok{19}\NormalTok{)}
\end{Highlighting}
\end{Shaded}

\begin{verbatim}
## 
##  One Sample t-test
## 
## data:  emiciones$azufre
## t = -0.32359, df = 39, p-value = 0.748
## alternative hypothesis: true mean is not equal to 19
## 95 percent confidence interval:
##  16.87912 20.53588
## sample estimates:
## mean of x 
##   18.7075
\end{verbatim}

\begin{Shaded}
\begin{Highlighting}[]
\CommentTok{\#t = {-}0.32359, df = 39, p{-}value = 0.748}

\CommentTok{\#valor de p{-}{-}{-}{-}{-}{-}{-}{-}{-}{-}{-}{-}{-}{-}{-}{-}{-}{-}{-}{-}{-}{-}{-}{-}{-}{-}{-}{-}{-}{-}{-}{-}{-}{-}{-}{-}{-}{-}{-}{-}{-}{-}{-}{-}{-}{-}{-}{-}{-}{-}{-}{-}{-}{-}{-}{-}{-}{-}{-}{-}{-}{-}{-}{-}}
\CommentTok{\#p{-}value = 0.748}

\CommentTok{\# Resultado {-}{-}{-}{-}{-}{-}{-}{-}{-}{-}{-}{-}{-}{-}{-}{-}{-}{-}{-}{-}{-}{-}{-}{-}{-}{-}{-}{-}{-}{-}{-}{-}{-}{-}{-}{-}{-}{-}{-}{-}{-}{-}{-}{-}{-}{-}{-}{-}{-}{-}{-}{-}{-}{-}{-}{-}{-}{-}{-}{-}{-}{-}{-}}

\CommentTok{\#el valor de las emiciones de óxido de azufre registradas si son significativamente mayores a los valores registrados por la empresa }
\CommentTok{\#valor de la empresa 17.5 T/año}
\CommentTok{\#valor calculado 18.70 T/año}
\end{Highlighting}
\end{Shaded}


\end{document}
